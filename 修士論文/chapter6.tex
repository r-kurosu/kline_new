\chapter{計算実験}\label{computational_result}
\subsection{計算時間の短縮}
\ref{近傍操作の計算時間短縮}で解の精度を変化させることなく計算時間を短縮する手法を提案した.
本説では,実際に計算実験を行い計算時間の比較を行う.

\begin{table}[]
  \centering
  \caption{近傍操作に関する計算時間の比較}
  \label{shift}
\begin{tabular}{cccrr}
\hline
注文数 & 積み地 & 揚げ地 & \multicolumn{1}{c}{\begin{tabular}[c]{@{}c@{}}全ての位置\\ を探索\end{tabular}} & 提案手法 \\ \hline
109 & 2   & 3   & 145                                                                      & 81                        \\
109 & 2   & 5   & 192                                                                      & 89                        \\
109 & 4   & 3   & 122                                                                      & 99                        \\
250 & 2   & 3   & 387                                                                      & 204                       \\ \hline
\end{tabular}
\end{table}

挿入近傍操作を一定回数行った際の計算時間の比較結果を表\ref{shift}に示す.
提案手法により、全てのインスタンスにおいて計算時間が短縮できることを確認した.

\subsection{MIPモデルとの比較}
元のモデルで探索を行って1時間後と24時間後の結果と,新たに提案するモデルの結果の比較を表\ref{heuristic}に示す.

結果より,いくつかのインスタンスでは計算時間を短縮しつつ,ある程度の精度の解を得ることができていることが確認できた.
その一方で,いくつかのインスタンスでは,局所探索で一度も実行可能解を得ることができなかった.
