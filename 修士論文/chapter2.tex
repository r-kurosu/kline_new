\chapter{問題設定}\label{definition}

複数の港を経由し, 自動車を輸送する運搬船を想定する. 例えば乗用車100台を港Aから港B, トラック30台を港Cから港Dというような各注文を, 運搬船の階層毎に一定の広さで区切られたスペースへの割当を計画する.この作業を席割作業(stowage plan)\cite{stowage}という. 本稿では複数の種類がある運搬船のなかで,単純な船の構造データを用いて実験する. また実際の席割業務では積載する自動車の種類が多様であり, ブルードーザーやショベルカーなどの建機は高さや重さが乗用車とは大きく異なるので特定の領域にしか収容出来ないという問題がある. これに対し本稿では作業自動化の初期段階として,自動車を全て乗用車とし, 船内の任意領域においてどの自動車も積載が可能と言う条件で席割を作成する. この章では一般的な船への自動車積載における専門用語の定義, 席割作業における入力情報や出力情報の説明をする.

\section{用語定義}
本研究で扱う船の航海等に関する専門用語の定義をする.

\begin{itemize}

\item 席割 \\
注文一つ一つを船のホールド割り当てる作業.

\item シミュレーション \\
席割で決まった自動車を一台ずつホールド内の領域に貼り付ける作業.

\item  プランナー \\
席割やシミュレーションを考える作業者.

\item  ギャング\\
港で自動車をホールドまで運転して積み降ろしをする作業者.

\item デッキ \\
船の内部の階層.

\item ホールド \\
各デッキ内を一定間隔の領域で仕切られた空間.

\item ランプ \\
上下デッキに移動するために各デッキの特定ホールドについているスロープ.

\item 注文 \\
乗用車100台を港Aから港Bへ輸送, トラック30台を港Cから港Dへ輸送というような積載自動車の情報とそれらの積み地と揚げ地に関する情報.

\item 積み地 \\
注文における自動車を積む港.

\item 揚げ地 \\
注文における自動車を降ろす港.

\item  RT (revenue ton) \\
基準となる車一台の面積に対する各注文の車の面積の割合. 本研究ではこれを資源要求量として扱う$[4]$.

\item ユニット数 \\
各注文に含まれる自動車の台数.

\end{itemize}

%2ページ

\section{入力情報}
本研究で扱う二種類の情報について述べる.
\begin{itemize}

\item 注文情報 \\
本稿では注文内に含まれる情報のうち注文番号, 積み地と揚げ地, ユニット数, 積載自動車のRT, 自動車の重量情報を扱う.

\item 船体情報 \\
輸送に使う船のデッキやホールド番号, 各ホールドへアクセスすために通るホールドや各ホールドの許容収容量などの船の内部構造に関する情報.

\end{itemize}

入力情報のうち注文情報の例を表\ref{table21}に, 船体情報の例を表\ref{table22}に示す. \\

\begin{table}[htbp]
\tabcolsep = 15pt
\renewcommand{\arraystretch}{0.8}
\caption{注文情報の例}
\label{table21}
\begin{center}
\begin{tabular}{rccrrr} \hline
注文番号 & 積み地 & 揚げ地 & ユニット数 & RT & 重さ(kg) \\ \hline
1 & 港1 & 港3 & 20 & 1.43 & 1480 \\
2 & 港1 & 港3 & 200 & 1.51 & 2680 \\
3 & 港1 & 港4 & 50 & 1.57 & 2580 \\
4 & 港2 & 港3 & 14 & 1.67 & 1720 \\
5 & 港2 & 港4 & 150 & 1.18 & 1810 \\
\hline
\end{tabular}
\end{center}
\end{table}

\begin{table}[htbp]
\centering
\tabcolsep = 15pt
\renewcommand{\arraystretch}{0.8}
\caption{船体情報の例}
\label{table22}
\begin{center}
\begin{tabular}{rrrr} \hline
デッキ番号 & ホールド番号 & 収容量 \\ \hline
1 & 2 & 80  \\
1 & 3 & 100  \\
2 & 1 & 25  \\
2 & 2 & 100 \\
2 & 3 & 150  \\
3 & 1 & 30 \\
3 & 2 & 200 \\
3 & 3 & 200 \\
\hline
\end{tabular}
\end{center}
\end{table}

%3ページ

\newpage

\section{出力}
入力情報を元に様々な条件を考慮して注文番号1番を2デッキ2ホールドに30台中30台割り当てるというような席割を出力する. 出力結果の例を表\ref{table23}に示す. \\

\begin{table}[htbp]
\centering
\tabcolsep = 15pt
\renewcommand{\arraystretch}{0.8}
\caption{出力結果の例}
\label{table23}
\begin{center}
\begin{tabular}{rcccrr} \hline
注文番号 & 積み地 & 揚げ地 & 割当先ホールド & 積載台数 & ユニット数 \\ \hline
1 & 港1 & 港3 & 3階-1 & 30 & 30 \\
2 & 港1 & 港3 & 2階-2 & 40 & 200 \\
2 & 港1 & 港3 & 3階-2 & 60 & 200 \\
2 & 港1 & 港3 & 3階-3 & 60 & 200 \\
3 & 港1 & 港4 & 2階-3 & 50 & 50 \\
4 & 港2 & 港3 & 2階-1 & 14 & 14 \\
5 & 港2 & 港3 & 1階-2 & 67 & 150 \\
5 & 港2 & 港4 & 1階-3 & 83 & 150 \\
\hline
\end{tabular}
\end{center}
\end{table}

%4ページ
