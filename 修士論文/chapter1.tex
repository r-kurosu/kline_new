\chapter{はじめに}
本研究では.複数の港で荷物を船に積み, 複数の港で降ろす運搬船を考える. 一般的には燃料や原料など多岐に渡る積荷を扱うが,本研究では乗用自動車を運搬する船を考える. 一般的に自動車運搬船は, 自動車を船の一定間隔で区切られたホールドと呼ばれるスペースにどの自動車を何台割り当てるかを決定する席割作業と呼ばれる工程を経て, その後席割作業で割り当てられた自動車に対して向きや場所を考慮して一台ずつ船内の領域に配置するシミュレーションと呼ばれる作業を行う. 現状自動車を輸送する会社はこの作業を人手で行っているが少なくとも席割作業に3時間, シミュレーション作業に4時間かかることから, 多大な人件費と時間をかける必要があり直前の追加注文等に対応が出来ないという問題点がある\cite{mitsui}. 本研究では席割作業とシミュレーション作業の2つの中で,席割作業の自動化に対して検証した結果を記す.

席割作業に関しては,斉藤,柳田\cite{saito}が自動車船に対する積み付け計画を行うアルゴリズムを提案している. また,斉藤らは同年に席割に関する初期解生成,解の探索に関して特許を取得している\cite{tokkyo}.
この特許は全ての車両が搬入可能であるための席割を作成するためのシステムであり,本研究で扱う複数の目的関数などを考慮していない.

本研究で扱う席割作業の概要について述べる. 様々な種類の車をA港からB港まで輸送するというような積載自動車の注文リストを受け取る. 注文リストを受け取ったプランナーと呼ばれる席割を作成する作業者は好ましい席割になるように, 注文の船内スペースへの割当を考える. 席割作成における前提条件について説明する. 例えば船内の特定の領域に積まれている自動車よりも奥に積まれている自動車が, あるC港で降ろされて空きスペースが発生したとする. この場合には,次のD港で積む自動車があればその自動車の積みやすさのために, 船内に積んだ自動車を奥側に詰めて手前の領域を確保するというような既に積まれた自動車の航海中の移動は原則しない. また自動車は全てのホールドに容易にアクセスすることは出来ない. 例えば本稿の実験で扱う自動車運搬船は12階構造の内5階のみに外部から自動車を入れるスロープがあり, 船内の奥側ホールドにアクセスしたい場合はそのホールドにアクセスする際に通過するホールドに十分な空きスペースがないとアクセスすることが出来ない.

本稿では第2章で席割問題に対する詳細な問題設定や, 本研究で扱う自動車輸送航海における専門用語について定義する.  第3章では注文に含まれる自動車一つ一つをどの領域に割り当てるかを最適化する数理モデルと, そのモデルに登場する本研究独自の制約や好ましい席割作成のための目的関数を説明する. 第4章では提案した数理モデルを商用ソルバーを用いて計算を行った際に求解が可能な問題例の規模の推察を行う. 
第5章ではヒューリスティックを用いる数理モデルを提案し, 第6章で実際の過去の航海データを基に求解実験を行い,比較実験を行う.
