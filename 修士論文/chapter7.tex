\chapter{まとめ}\label{conclution}
自動車運搬船における自動車を船に割り当てる席割の作成に関して,プランナーやオペレータとのヒアリングを基に定式化を行い2つの数理モデルを提案した.

整数計画ソルバーを用いて問題を解く際に,有効な解を出力することのできる問題例の規模を確認した.また,注文数が増えた場合や更なる制約や目的関数を追加する場合に計算時間が膨大に増大しないことを目指し,ヒューリスティックを用いた新たな数理モデルを提案した.

新たに提案した数理モデルの局所探索における近傍操作に関して,挿入近傍操作において解の精度を落とすことなく計算時間を短縮する手法を提案し,有効性を確認した.
また,局所探索における初期解を生成する手法を2種類提案し,計算実験を行い比較を行った..

ヒューリスティックを基にした新たな数理モデルを用いることで,整数計画問題を解く手法と比較して,計算時間の短縮及びある程度の精度の解を得ることができることを確認した.
