\chapter{提案手法1}\label{method1}

\label {gurobiの計算実験}
鵜川ら\cite{ukawa}は,第3章で定式化した整数計画問題に対して,整数計画ソルバーを用いて実行可能解を得ることを提案した.

計算実験の結果,規模の小さい問題例に対しては有効な解を得ることを確認したが,注文数や積み地や揚げ地の数の増加とともに制約や組合せが急速に増加してしまい,指定された時間内では十分に計算が進まない可能性が指摘されている.

\section{求解が可能な問題例の規模}
\label{mip}
注文数や港の数を変更した問題例をいくつか準備し,どの程度の規模の問題例ならばある程度の時間内で解けるのか調査を行った.
実験に用いるプログラムは Pythonを用いて実装し, 計算機はPowerEdge T320 (CPU: Intel(R) Xeon(R) CPU E5-1410 v2 (2.80 GHz, 10 M cache), RAM: 96 GB) を使用した. また,整数計画ソルバーとしてGurobi Optimizer (ver 9.0.0) を使用した.

表\ref{24hours}に実行可能解が初めて出力された計算時間と,計算を回して1時間後と24時間後の解の精度を示した.

結果より,注文数が250程度,港の数が3,4より少なければ,24時間以内にある程度の解を出力することが確認できた.
一方で,注文数が350の問題例に関しては,24時間後でも解の探索があまり進まなかった.

実際にはこれ以上の注文数を持つ問題例が多く存在しており,整数計画ソルバーを用いる手法では,より大規模な問題例を解くことは難しいと考えられる.

\begin{landscape}
\begin{table}[]
  \centering
  \caption{複数の規模の問題例に対する計算結果}
  \label{24hours}
  \begin{tabular}{cccrrrrr}
  \hline
  \multicolumn{3}{c}{問題例} & \begin{tabular}[c]{@{}c@{}}実行可能解が\\ 出力された時間\end{tabular} & \multicolumn{1}{c}{\begin{tabular}[c]{@{}c@{}}1時間後の\\ 解\end{tabular}} & \multicolumn{1}{c}{\begin{tabular}[c]{@{}c@{}}1時間後の\\ 下界値\end{tabular}} & \multicolumn{1}{c}{\begin{tabular}[c]{@{}c@{}}24時間後の\\ 解\end{tabular}} & \multicolumn{1}{c}{\begin{tabular}[c]{@{}c@{}}24時間後の\\ 下界値\end{tabular}} \\ \hline
  注文数   & \begin{tabular}[c]{@{}c@{}}積み地\\ の数\end{tabular}   & \begin{tabular}[c]{@{}c@{}}揚げ地\\ の数\end{tabular} &                                                       & \multicolumn{1}{l}{}                                                  & \multicolumn{1}{l}{}                                                    & \multicolumn{1}{l}{}                                                   & \multicolumn{1}{l}{}                                                     \\ \hline
  109   & 2       & 3       & 537                                                   & $-4301.36      $                                                         & $-4461.98    $                                                             & $-4388.44    $                                                            & $-4461.98 $                                                                 \\
  109   & 2       & 5       & 1674                                                  & $-3522.82                                          $                     & $-4461.98        $                                                         & $-4264.69   $                                                             & $-4461.98 $                                                                 \\
  109   & 4       & 3       & 344                                                   & $-3515.46                                           $                    &$ -4461.98       $                                                          &$ -4365.89  $                                                              & $-4461.98 $                                                                 \\
  250   & 2       & 3       & 2381                                                  & $-3155.02                                $                               & $-4461.98        $                                                         & $-4083.18  $                                                              &$ -4461.98   $                                                               \\
  350   & 2       & 3       & 2249                                                  & $ -2738.00                   $                                            &$-4461.98             $                                                    & $-2883.58                    $                                            & $-4461.98                   $                                               \\ \hline
  \end{tabular}
\end{table}
