\chapter{MIPを用いた計算実験}\label{method}

\label {gurobiの計算実験}
鵜川らによる研究の数理モデルでは自動車の注文数や積み地や揚げ地の数の増加とともに制約や組合せが急速に増加してしまい,指定された時間内では十分に計算が進まない可能性が指摘されている\cite{ukawa}.

\subsection{解くことのできる問題例の規模}
注文数や港の数を変更した問題例をいくつか準備し,どの程度の規模の問題例ならばある程度の時間内で解けるのか調査を行った.
実験に用いるプログラムは Pythonを用いて実装し, 計算機はPowerEdge T320 (CPU: Intel(R) Xeon(R) CPU E5-1410 v2 (2.80 GHz, 10 M cache), RAM: 96 GB) を使用した. また,整数計画ソルバーとしてGurobi Optimizer (ver 9.0.0) を使用した.

表\ref{24hours}に実行可能解が初めて出力された計算時間と,計算を回して1時間後と24時間後の解の精度を示した.

結果より,注文数が250程度,港の数が3,4より少なければ,24時間以内にある程度の解を出力することが確認できた.
一方で,注文数が350の問題例に関しては,24時間後でも解の探索があまり進まなかった.

実際にはこれ以上の注文数を持つ問題例が多く存在しており,提案したモデルではより大規模な問題例を解くことは難しいと考えられる.
