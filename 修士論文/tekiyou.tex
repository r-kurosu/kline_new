\begin{center}
{\LARGE 自動車運搬船における \\ 貨物積載プランニングの席割問題}\\[0.5cm]
\end{center}
\hfill
{\large 252001054\qquad 竹田 陽}\\[0.5cm]
\begin{center}
{\Large \bf 概 要}\\
\end{center}

ある港から複数の港を経由して自動車を積み, 複数の港で自動車を降ろす自動車運搬船について考える. 自動車運搬船に積む自動車の集合が与えられてから実際に自動車が船に積まれるまでに, 席割とシミュレーションと呼ばれる二種類の作業が行われる. 席割では船の階層内を一定間隔の大きさに区切ったホールドと呼ばれるスペースの各々に, 与えられた積載自動車リストのどの自動車を何台割り当てるかを考える. シミュレーションでは席割作業でホールド毎に割り当てられた自動車を, 向きと場所考慮してホールド内に配置する. 本研究ではこの席割作業とシミュレーション作業の全自動化への第一歩として, 席割作業を自動化するために数理最適化の技術によってコンピュータで短時間かつ効率の良い席割を出力することを目標とする.

席割問題は割当型の問題であるが, 特徴的な要件として以下の2つがある. 一点目は自動車を船に積むあるいは船から降ろす際に, 人が自動車を運転して所定の位置まで移動するための通路を確保する必要があることである. このとき既に積んだ自動車が新たに積みたいあるいは降ろしたい自動車の進行経路上にあると, 自動車は割り当てられたホールドと船の出入り口間を移動することができない. 二点目は船全体の荷重バランスを考慮しなければならないことである. 例えば船が海上を進む際に積まれた自動車が船の一部の領域に集中していると, 船が波に揺られたときにそのまま横転してしまう可能性があり非常に危険である.

良い席割の指標について, 海運会社の業務として席割を作成している作業者や自動車を船内のエリアまで運転して運ぶ運転手へのヒアリングの結果, 積み降ろし時に積み降ろす自動車を間違えることが起こりにくいこと, 運転手が自動車を運転しやすいこと, 船のスペースを無駄にしないことに重点を置いていると考えられた. 本研究ではこのような良い席割の指標を考慮した目的関数を設計する. 
