\begin{center}
{\LARGE 自動車運搬船における \\ 貨物積載プランニングの席割問題}\\[0.5cm]
\end{center}
\hfill
{\large 252001054\qquad 竹田 陽}\\[0.5cm]
\begin{center}
{\Large \bf 概 要}\\
\end{center}

ある港から複数の港を経由して自動車を積み, 複数の港で自動車を降ろす自動車運搬船における貨物積載について考える. 自動車運搬船に積む自動車の情報が与えられてから実際に自動車が船に積まれるまでに, 席割とシミュレーションと呼ばれる二種類の作業が行われる. 席割作業では船の階層内を一定間隔の大きさに区切ったホールドと呼ばれるスペースに, 与えられた積載自動車リストのどの自動車を何台割り当てるかを決める. シミュレーション作業では席割作業でホールド毎に割り当てられた自動車を, 自動車の向きや空きスペース,作業効率などを考慮してホールド内への配置を決定する. 本研究では,席割作業を自動化するために数理最適化の技術によってコンピュータで短時間かつ効率の良い席割を出力することを目標とする.

席割問題は割当問題として定式化することができるが, 特徴的な要件として以下の2つがある. 一点目は自動車を船に積むあるいは船から降ろす際に, 人が自動車を運転して所定の位置まで移動するための通路を確保する必要があることである. このとき既に積んだ自動車が,新たに積むもしくは降ろす自動車の進行経路上にあると, 自動車は割り当てられたホールドと船の出入り口間を移動することができない. 二点目は船全体の荷重バランスを考慮しなければならないことである. 例えば船が海上を進む際に積まれた自動車が船の一部の領域に集中していると, 船が波に揺られたときにそのまま横転してしまう可能性があり非常に危険である.

良い席割の指標について, 海運会社の業務として席割を作成している作業者や自動車を船内のエリアまで運転して運ぶ運転手へのヒアリングを行った.ヒアリングから,積み降ろし時に意図しない自動車を積み下ろすことが起こりにくいこと, 積み降ろしをする港が同じ自動車が船内でまとまって割り当てられていること,運転手が作業をする効率が落ちない程度のスペースが確保されていること, 船のスペースを無駄にしないことに重点を置いていると考えられた. 本研究ではこのような良い席割の指標を考慮した目的関数を設計する.

本研究では,良い席割を作成するための指標や特徴を考慮した数理モデルを提案する.商用の整数計画ソルバー (Gurobi Optimizer) を使用した厳密解法に基づく数理モデルと,局所探索法を用いた近似解法に基づく数理モデルの2種類を提案し,計算時間や精度の比較を行った.計算実験の結果,局所探索法を用いたモデルを用いることで,解の精度をあまり落とすことなく計算時間を短縮できることを確認した.また,局所探索における初期解生成に関して,2種類のアプローチを提案する.一つ目は,整数計画問題として定式化したモデルの線形緩和問題を解き,その解を初期解として利用するというものである.二つ目は,積む港と降ろす港が同じ注文をまとめて一つとみなし,整数計画問題を解き解を初期解として利用するというものである.提案した2種類のアプローチと,ランダムに生成した解の3種類を局所探索における初期解として探索を行い,局所探索が終了するまでの計算時間と解の精度の比較を行う.
